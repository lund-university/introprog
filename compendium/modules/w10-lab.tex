%!TEX encoding = UTF-8 Unicode

%!TEX root = ../compendium.tex

\Lab{\LabWeekTEN}

\begin{Goals}
\item Kunna använda inbyggda sorteringsfunktioner.
\item Kunna använda inbyggda sökfunktioner.
\item Känna till hur strängar ordnas.
\item Kunna läsa text i tabellform från fil.
\item Kunna använda registrering (frekvensräkning) för enkla statistikberäkningar.
\item ... \TODO mer här
\end{Goals}

\begin{Preparations}
\item \StudyTheory{10}
\item \DoExercise{\ExeWeekTEN}{10}
\item \ReadTheLab
\item Svara på denna enkät $<<$ \TODO Länk till google forms-enkät $>>$  \\
\TODO förslag till Anton på innehåll i google forms-enkät:\\ \textit{Vilket är ditt favoritalternativ?}
\begin{itemize}[nolistsep,noitemsep]
\item \textbf{program} (D, W, C, E, F, I, Bio, K, L, M, Bme, Nano, V), 
\item \textbf{OS} (Win7, Win10, macOS X, Linux, Android, IOS, ChromeOS), 
\item \textbf{editor} (gedit, vim, emacs, vi, notepad++, sublime text, atom)
\item \textbf{IDE} (Eclipse, IntelliJ/AndroidStudio, VisualStudio, xcode), 
\item \textbf{socialnät} (facebook, snapchat, linkedin, instagram, github), 
\item \textbf{webbläsare} (firefox, chrome, safari, edge, vivaldi, opera)
\item \textbf{sorteringsalgoritm} (insättningssortering, urvalssortering)
\item \textbf{språk} (Java, Python, PHP, C\#, Javascript, C++, C, Objective-C, R, Swift, Matlab, Ruby, Visual Basic, VBA, Scala, Perl, lua, Delphi)  \\
Listan ordnad enligt \url{http://pypl.github.io/PYPL.html} i Aug 2016
% Alternativet är TIOBE, men den är längre...:
%(Java, C, C++, C\#, Python, PHP, Javascript, Visual Basic, Perl, Pascal, Ruby, Swift, Groovy, R, Matlab, SQL, Go, Dart, Fortram, Lua, Ada, Lisp, Scala, Prolog, Haskell, Erlang, Rust)
\end{itemize}
\end{Preparations}


\subsection{Bakgrund}

I denna laboration ska du utveckla ett program som analyserar svar på enkäter med flervalsfrågor. Indata utgörs av text i form av \textbf{kolumnseparerade värden}, där varje persons svar finns på en egen rad och varje svarsrad innehåller svarsalternativ separerade med en \textbf{kolumnseparator} som till exempel kan vara \code{;} eller \code{\t}. Första raden i textfilen anger kolumnernas namn.

Exempel på indatafil: \footnote{\TODO gör indataexemplet lite längre så analysexempel blir lite roligare och låt indatafilen finnas med som testfil i workspace}
\begin{CodeSmall}[language=, ]
program;OS;editor;IDE;socialnät;webbläsare;sorteringsalgoritm;språk
W;Gedit;Eclipse;Facebook;Firefox;insättningssortering;Java
D;Atom;Intellij;GitHub;Vivaldi;urvalssortering;Scala
E;emacs;Eclipse;Snapchat;Edge;urvalssortering;C
\end{CodeSmall}

Ditt program ska innehålla följande delar:
\begin{itemize}
\item En case-klass för strängmatriser 
\item Funktioner för inläsning av tabellformatterad text.
\item Funktioner för att presentera statistik från enkätdata med hjälp av registrering.
\item \TODO mer här ?
\end{itemize}

\TODO mer här...

\subsection{Obligatoriska uppgifter}

\Task En labbuppgiftsbeskrivning.

\Subtask En underuppgift.

\Subtask En underuppgift.

\subsection{Frivilliga extrauppgifter}
    
\Task En labbuppgiftsbeskrivning.

\Subtask En underuppgift.

\Subtask En underuppgift.

