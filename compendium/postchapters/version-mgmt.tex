%!TEX encoding = UTF-8 Unicode
%!TEX root = ../compendium.tex


\chapter{Versionshantering och kodlagring}

\section{Vad är versionshantering?}

\textbf{Versionshantering}\footnote{\href{https://en.wikipedia.org/wiki/Version_control}{en.wikipedia.org/wiki/Version\_control}} \Eng{version control eller revision control} av mjukvara innebär att hålla koll olika versioner av koden i ett utvecklingsprojekt allteftersom koden ändras. Versionshantering är en deldisciplin inom \textbf{konfigurationshantering} \Eng{software configuration managament} som inbegriper allt som rör processen för att identifiera, besluta, genomföra och följa upp ändringar i koden.

En viktig del av versionshantering är att lagra olika versioner av koden allt eftersom den utvecklas. Ett bra verktygsstöd och en väldefinierad arbetsprocess för versionshanteringen, som alla i utvecklingsprojektet följer, möjliggör att flera utvecklare kan arbeta parallellt med att göra tillägg och ändringar i samma kodbas utan att skapa kaos och förvirring för varandra.

God versionshantering är helt avgörande för utvecklarnas produktivitet, speciellt för stora projekt med många utvecklare som jobbar parallellt mot en omfattande kodbas med många olika interna och externa komponenter, till exempel öppenkällkod. 

Men även i ett litet projekt med endast en utvecklare kan man ha god nytta av ett versionshanteringsverktyg och ett disciplinerat förfarande för att namge versioner, t.ex. för att kunna återskapa tidigare versioner av projektets olika kodfiler när en ändring visade sig mindre lyckad.   

Det finns olika modeller för versionshanteringen:
\begin{itemize}
\item \textbf{lokal}; alla utvecklare jobbar i samma, lokala filsystem där alla olika versioner lagras.
\item \textbf{centraliserad}; ett repositorium (förk. repo), alltså en databas med koden, finns centralt på en server som alla jobbar mot med hjällp av en versionshanteringsklient.
\item \textbf{distribuerad}; alla utvecklare har sitt eget lokala repo och varje utvecklare initierar enskilt när ändringar ska delas mellan olika repo.
\end{itemize}


\section{Versionshanteringsverktyget git}

Det finns många versionshanteringsverktyg\footnote{\href{https://en.wikipedia.org/wiki/List_of_version_control_software}{https://en.wikipedia.org/wiki/List\_of\_version\_control\_software}}, men på senare tid dominerar verktyget \texttt{git}, speciellt i i öppenkällkodsvärlden. \texttt{git} utvecklades ursprungligen av Linus Torvalds för att versionshantera Linuxkärnan, men har växt till ett omfattande öppenkällkodsprojekt med stor spridning och många användare.



\subsection{Installera git}

\subsection{Använda git}

\section{Vad är nyttan med en kodlagringsplats?}

\section{Kodlagringsplatsen GitLab}


\section{Kodlagringsplatsen GitHub}

\subsection{Installera klienten för GitHub}

\subsection{Använda GitHub}


\section{Kodlagringsplatsen BitBucket}

\subsection{Installera SourceTree}

\subsection{Använda SourceTree}
