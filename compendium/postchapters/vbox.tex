%!TEX encoding = UTF-8 Unicode
%!TEX root = ../compendium.tex

\chapter{Virtuell maskin}\label{appendix:vbox}

\section{Vad är en virtuell maskin?}

Du kan köra alla kursens verktyg i en så kallad virtuell maskin (vm). Det är ett enkelt och säkert sätt att installera ett nytt operativsystem i en ''sandlåda'' som inte påverkar din dators ursprungliga operativsystem. 

\section{Installera kursens vm}
Det finns en virtuell maskin förberedd med alla verktyg som du behöver förinstallerade. Gör så här:
\begin{enumerate}
\item     Installera VirtualBox v5 här: \\ \url{https://www.virtualbox.org/wiki/Downloads}
\item     Ladda ner filen vbox.zip här: \\ \url{http://fileadmin.cs.lth.se/pgk/vbox.zip} \\ OBS! Då filen är på nästan 4GB kan nedladdningen ta mycket lång tid.
\item     Packa upp filen vbox.zip i biblioteket "VirtualBox VMs" som du fick i din hemkatalog när du installerade VirtualBox. Du får då 3 filer som heter något med "introprog-ubuntu-64bit".
\item     Kolla med hjälp av denna sida: \\ \url{https://md5file.com/calculator} \\ så att filen "introprog-ubuntu-64bit.vdi" har denna sha256-cheksumma: \\ --- ska-stå-checksumma-här-sen ---
\item     Öppna VirtualBox och lägg till maskinen introprog-ubuntu-64bit genom menyn ''add''.
\item     Starta maskinen.
\item     Öppna ett terminalfönster och skriv scala och du är igång och kan göra första övningen!
\end{enumerate}

\section{Vad innehåller kursens vm?}

Den virtuella maskinen kör Xubuntu 14.04 med fönstermiljön XFCE, vilket är samma miljö som E-husets linuxdatorer kör. 

I den virtuella maskinen finns detta förinstallerat:

\begin{itemize}
\item Java JDK 8
\item Scala 2.11.8
\item Kojo 2.4.08
\item Eclipse Mars.2 med ScalaIDE 4.3
\item gedit med syntaxfärgning för Scala och Java
\item git
\item sbt
\item Ammonite REPL
\end{itemize}