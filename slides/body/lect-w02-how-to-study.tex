%!TEX encoding = UTF-8 Unicode
%!TEX root = ../lect-week02.tex

\Subsection{Kort om förra veckan och kommande två veckor}

\begin{Slide}{Förra veckan}
Viktiga övergripande mål:
\begin{itemize}
\item Förstå skillnaden mellan värde och typ
\item Börja använda sekvens, alternativ, repetition, abstraktion
\item Förstå variabel och tilldelning
\item Reflektera över ditt lärande
\item Träffas i samarbetsgrupper och lära känna varandra
\item Komma in i kursens arbtesgång: förel. -> övn. -> labb
\end{itemize}
Om du inte hann klart labben, fortsätt på egenhand och på resurstid. Avsätt mer tid till labbförberedelser nästa gång.
\end{Slide}


\begin{Slide}{Kommande 2 veckor}
Övergripande mål:
\begin{itemize}
\item Börja skriva dina \Emph{egna program}
\item Träna på att dela upp din kod i \Alert{många små funktioner}
\end{itemize}

Läsvecka 2:
\begin{itemize}
\item Övning \texttt{\ExeWeekTWO}
\item Ingen labb.
\end{itemize}
Läsvecka 3:
\begin{itemize}
\item Övning \texttt{\ExeWeekTHREE}
\item Labb \texttt{\LabWeekTHREE}  \\ spela varandras textspel i samarbetsgrupper
\end{itemize}

\vspace{1em}\Alert{OBS!} Noter schemaavvikelser i läsvecka 3 och vecka 5.\\\url{http://cs.lth.se/pgk/schema/timeedit/}

\end{Slide}



\Subsection{Studieteknik}

\begin{Slide}{Hur studerar du?}
\begin{itemize}
\item Vad är bra \Emph{studieteknik}?
\item Hur lär \Alert{du} dig bäst? Olika personer har olika preferenser.
\begin{itemize}
\item Ta reda på vad som funkar bäst för dig.
\item En kombination av flera sinnen är bäst: läsa+prata+skriva...
\item Aktivera dig! Inte bara passivt läsa utan också aktivt göra.
\end{itemize}

\item Hur skapa \Emph{struktur}?
Du behöver ett sammanhang, ett \Emph{system av begrepp}, att \Alert{placera in} din nya kunskap i.
\item Hur uppbåda \Emph{koncentration}? Steg 1: Stäng av mobilen!
\item Hur vara \Emph{disciplinerad}? Studier först, nöje sen!
\item Du måste \Emph{planera och omplanera} för att säkerställa \Alert{tillräckligt mycket egen pluggtid} då du är pigg och koncentrerad för att det ska funka!
\item Programmering \Alert{kräver} en \Emph{pigg och koncentrerad hjärna}!
\end{itemize}
\end{Slide}


\begin{Slide}{Hur ska du studera programmering?}
\begin{itemize}
\item När du gör \Emph{övningarna}:
\begin{itemize}
\item Ta fram föreläsningsbilderna i pdf och läs igenom dem \Alert{innan} du påbörjar övningen.
\item Är det något i föreläsningsbilderna du inte förstår: ta upp det i samarbetsgrupperna eller på resurstiderna.
\item Om något är knepigt:
\begin{itemize}
\item Hitta på egna REPL-experiment och undersök hur det funkar.
\item Följ ev. länkar i föreläsningsbilderna, eller googla själv på wikipedia, stackoverflow, ...
\end{itemize}
\end{itemize}

\item Innan du gör \Emph{laborationerna}:
\begin{itemize}
\item Gör minst grundövningarna \Alert{innan} du börjar med labben. Dubbelkolla att du har uppnått övningsmålen.
\item Läs igen \Alert{hela} labbinstruktionen \Alert{innan} labben och gör en bedömning av din förberedelsetid.
\item Gör förberedelserna \Alert{i god tid innan} labben.
\item Om du tror att du behöver det för att hinna med: \\ gör delar av labben \Alert{innan} ditt labbtillfälle.
\end{itemize}
\end{itemize}

\end{Slide}

\begin{Slide}{Det går inte att förstå allt på en gång!}
\begin{itemize}
\item Vi nosar på ett visst begrepp på ytan i en vecka ...

\item ... för att i senare vecka återkomma till det, men djupare.

\item Förståelse kommer efter hand och kräver bearbetning.

\item Vi måste \Emph{iterera} begreppen innan vi kan nå djup.

\pause\item Det är svårt för dig nu att se vad som är \Alert{detaljer} som du inte ska hänga upp dig på, och vad som är \Emph{det viktiga} i detta läget. Men det kommer! Ha tålamod!
\end{itemize}

\end{Slide}


\begin{Slide}{På rasten: träffa din samarbetsgrupp}
\begin{itemize}
\item Träffas i samarbetsgrupperna och bestäm/gör/diskutera:
\begin{enumerate}
\item När ska ni träffas nästa gång?
\item Bläddra igenom föreläsningsbilderna från w01 i pdf.
\item Vilka \Emph{koncept} är fortfarande (mest) \Alert{grumliga}? \\Alltså: Vilka koncept från förra veckan vill ni på nästa möte jobba mer med i gruppen för att alla ska förstå grunderna?
\end{enumerate}
\end{itemize}
\vspace{1em}Du som ännu inte visat ditt \Alert{samarbetskontrakt}, visa det för handledare på \Emph{veckans resurstid}. Exempel på kontrakt:
\\\href{https://github.com/lunduniversity/introprog/tree/master/study-groups}{\footnotesize\texttt{github.com/lunduniversity/introprog/tree/master/study-groups}}
\end{Slide}


%%%
